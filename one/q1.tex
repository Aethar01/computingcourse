\documentclass[a4paper,english]{article}

%% Use utf-8 encoding for foreign characters
%%\usepackage[T1]{fontenc}
%%\usepackage[utf8]{inputenc}
%%\usepackage{babel}
%%
%%%% Vector based fonts instead of bitmaps
%%\usepackage{lmodern}
%%
%%%% Useful
%%%\usepackage{fullpage} % Smaller margins
%%\usepackage{enumerate}
%%
%%%% Theorem
%%\usepackage{amsthm}
%%
%%%% More math
%%\usepackage{amsmath}
%%\usepackage{amssymb}

%% Document Header
\title{Section1:Primer}
\author{Elliott Ashby}
\date{\today}

\begin{document}
    \maketitle
    \section{q1}
        The answers are m and 3 since the 4th letter of the 3rd to last element of the list b is m and the first entry in a is 3.
    \section{q2}
       The while loop multiplies fac by i returning it to itself then prints fac and i then increments
       i by one until i reaches 10. Essentially printing a pattern where the right number is multiplied by the 
       number on the next row on the left to find the next number on the right side. Hence \\
       \begin{center}
           \begin{tabular}{ |c|c| }
                1 & 1 \\
                2 & 2 \\
                3 & 6 \\
                4 & 24 \\
                5 & 120 \\
                6 & 720 \\
                7 & 5040 \\
                8 & 40320 \\
                9 & 362880
           \end{tabular}
       \end{center}
       Since i reaches 10 before it is multiplied by fac the while loop ends and a row with i as 10 does not get printed.
    \section{q5}
       $y^2$ and x converge, however, due to imprecise float calculations, python does not see them as equal, therefore the condition for the while loop never arrises causing an infinite loop.
    \section{q6}
       \begin{center}
           \begin{tabular}{ |c|c| }
               \input{./q1_6.txt}
           \end{tabular}
       \end{center}
    \section{q7}
        when truncating each step in a() we get a result of 0.01 \\ however when using b() and truncating each step we get a value of $5.00\times10^{-3}$, which is more accurate since the result from a() has a lower resolution.
        
\end{document}
