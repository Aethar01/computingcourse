\documentclass[a4paper,english]{article}
\usepackage{graphicx}
\usepackage{listings}
\usepackage{amsmath}
\usepackage{multirow}
%% Use utf-8 encoding for foreign characters
%%\usepackage[T1]{fontenc}
%%\usepackage[utf8]{inputenc}
%%\usepackage{babel}
%%
%%%% Vector based fonts instead of bitmaps
%%\usepackage{lmodern}
%%
%%%% Useful
%%%\usepackage{fullpage} % Smaller margins
%%\usepackage{enumerate}
%%
%%%% Theorem
%%\usepackage{amsthm}
%%
%%%% More math
%%\usepackage{amsmath}
%%\usepackage{amssymb}
\lstset{
  breaklines=true,
  postbreak=\mbox{{$\hookrightarrow$}\space},
  backgroundcolor = \color{lightgray},
}
%% Document Header
\title{Section5}
\author{Elliott Ashby}
\date{\today}

\begin{document}
    \maketitle
    \section{q1}
    \lstinputlisting[language=Python]{./q8_1.py}
    This code has 2 functions, neutron which returns a location between 0 and L and the number of nuetrons from the fission
    (in this case 2). And the other function is chain, which takes a location, number of neutrons and L. It then randomly 
    determines which direction the neutrons travel and excludes them if they are outside the range of 0 to L. It then
    returns the number of total nuetrons that stay within the range.
    \\
    \section{q2}
    \lstinputlisting[language=Python, firstline=6, lastline=9]{./q8_2.py}
    Here, we simply change the number of neutrons to be set by the neutrons.neutrons() function and not a static 2.
    \section{q3}
    \lstinputlisting[language=Python, firstline=32, lastline=45]{./q8_2.py}
    Here we update neutron to include the function neutrons which returns a random number with average of 2.5 and make 
    a new while loop in order to determine the critical value. It does this by taking an average count of secondary fissions
    of 1000 fissions each with 100 initial fissions. Once the average is larger than the inital fissions, it means more 
    secondary fissions occur than initial fissions.
    Using this we get a critical value:
    \begin{center}
        $L_{critical value} \approx 0.412 \pm 0.0005$
    \end{center}
    Using more than 100 inital neutrons simply gets a more precision average allowing for a more more precise answer...
    if the small increase in L allows. In our case increasing the initial fissions doesnt increase the precision of
    $L_{critical value}$.
    \section{q4}
    In order to implement 3 dimensions we need to modify both our functions but not our main.
    \lstinputlisting[language=Python, firstline=7, lastline=36]{./q8_4.py}
    Since python is dynamically typed we can simply change out loc to a list instead of a float. And in chain we can
    randomly determine a vector as a list and add it to the the location passed into the function.
    \\
    The final change is to the check whether the new value is in range. We simply expand this to check all dimensions are
    within 0 to L.
    \section{q5}
    Using the above code we can determine a critical value and hence the critical mass, using an increment of 0.001 and initial fissions of 100:
    \begin{center}
        $L_{criticalvalue} \approx 0.1478 \pm 0.00005$\\
        \begin{equation*}
            m_{critical} = L_{criticalvalue}^3 \times \rho_{Uranium}
        \end{equation*}
    Using $\rho_{Uranium} = 18.7\textrm{Mgm$^{-3}$}$:\\
        $m_{critical} \approx 60.38\textrm{kg}$
    \end{center}

\end{document}
