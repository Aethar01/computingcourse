\documentclass[a4paper,english]{article}
\usepackage{graphicx}
\usepackage{listings}
\usepackage{amsmath}
\usepackage{multirow}

%% Use utf-8 encoding for foreign characters
%%\usepackage[T1]{fontenc}
%%\usepackage[utf8]{inputenc}
%%\usepackage{babel}
%%
%%%% Vector based fonts instead of bitmaps
%%\usepackage{lmodern}
%%
%%%% Useful
%%%\usepackage{fullpage} % Smaller margins
%%\usepackage{enumerate}
%%
%%%% Theorem
%%\usepackage{amsthm}
%%
%%%% More math
%%\usepackage{amsmath}
%%\usepackage{amssymb}
\lstset{
  breaklines=true,
  postbreak=\mbox{{$\hookrightarrow$}\space},
}

%% Document Header
\title{Section1}
\author{Elliott Ashby}
\date{\today}

\begin{document}
    \maketitle
    \section{q1}
    The while loop multiplies fac by i returning it to itself then prints fac and i then increments
    i by one until i reaches 10. Essentially printing a pattern where the right number is multiplied by the 
    number on the next row on the left to find the next number on the right side. Hence \\
    \begin{center}
        \begin{tabular}{ |c|c| }
            1 & 1 \\
            2 & 2 \\
            3 & 6 \\
            4 & 24 \\
            5 & 120 \\
            6 & 720 \\
            7 & 5040 \\
            8 & 40320 \\
            9 & 362880
        \end{tabular}
    \end{center}
    Since i reaches 10 before it is multiplied by fac the while loop ends and a row with i as 10 does not get printed.
    \section{q2}
    \lstinputlisting[language=Python]{./1_2.py}
    Here we see a some code the loops over a print statement mutating the variable i each time, starting from 1 and going to 10.
    The print statement print the value, it's square and it's cube.
    \section{q3}
    The answers are m and 3 since the 4th letter of the 3rd to last element of the list b is m and the first entry in a is 3.
    \section{q4} 
    
    \section{q5}
    
\end{document}
